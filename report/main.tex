% Document configuration 

\documentclass{beamer}

\usepackage[utf8]{inputenc}
\usepackage{listings}
\usepackage{xcolor}

% Color definitions
\definecolor{codegreen}{rgb}{0,0.6,0}
\definecolor{codegray}{rgb}{0.5,0.5,0.5}
\definecolor{codepurple}{rgb}{0.58,0,0.82}
\definecolor{backcolour}{RGB}{240,240,240}

% Define a listing style
\lstdefinestyle{mystyle}{
  backgroundcolor=\color{backcolour}, commentstyle=\color{codegreen},
  keywordstyle=\color{magenta},
  numberstyle=\tiny\color{codegray},
  stringstyle=\color{codepurple},
  basicstyle=\ttfamily\footnotesize,
  breakatwhitespace=false,         
  breaklines=true,                 
  captionpos=b,                    
  keepspaces=true,                 
  numbers=left,                    
  numbersep=5pt,                  
  showspaces=false,                
  showstringspaces=false,
  showtabs=false,                  
  tabsize=2
}

% Set the listing style
\lstset{style=mystyle}

\usetheme{Madrid}

%--------------------------

% Title page setup

\title[Plataforma Digital]
{Logística Urbana para Entrega de Mercadorias}

\subtitle{Plataforma Digital}

\author[Grupo 30, 2LEIC03]
{Guilherme Sequeira, Pedro Ramalho, Tomás Pacheco}

\institute[FEUP]
{
  Faculdade de Engenheria\\
  Universidade do Porto
}

\date[2021/2022 2S]
{Desenho de Algoritmos, 2021/2022, 2º semestre}

%--------------------------

% Display the table of contents at the beginning of each section

\AtBeginSection[]
{
  \begin{frame}
    \frametitle{Conteúdos}
    \tableofcontents[currentsection]
  \end{frame}
}

%--------------------------

%----------------------------------------------

% START DOCUMENT

\begin{document}


% Initialize title page

\frame{\titlepage}


% Initialize table of contents

\begin{frame}
  \frametitle{Conteúdos}
  \tableofcontents
\end{frame}




%-------------------------------------------------------

% Begin section : Descrição do problema

\section{Descrição do problema}

%-------------------------------------------------------

% frame 1
\begin{frame}[fragile]
\frametitle{Sample frame title}
Descrição do problema goes here.  
\end{frame}

% End section : Descrição do problema

%-------------------------------------------------------




%-------------------------------------------------------

% Begin section : Cenário 1 - Minimização de Estafetas

\section{Cenário 1 - Minimização de Estafetas}

% frame 1
\begin{frame}
\frametitle{Formalização do Problema}
Super important text.
\end{frame}

% frame 2
\begin{frame}[fragile]
\frametitle{Descrição do Algoritmo}
Em baixo encontra-se pseudocódigo para o algoritmo usado no cenário 1:
\begin{lstlisting}[language=python]
for estafeta in estafetas:
  estafeta.valor = estafeta.max_vol + estafeta.max_peso

for entrega in entregas:
  entrega.valor = entrega.vol + entrega.peso

sort(estafetas, por valor, ordem decrescente)
sort(entregas, por valor, ordem decrescente)

for entrega in entregas:
  for estafeta in estafetas:
    if fits(entrega, estafeta):
      estafeta.add_entrega(entrega)
      break
\end{lstlisting}
\end{frame}

% frame 3
\begin{frame}
  \frametitle{Análise da Complexidade}
  Super important text.
\end{frame}

% frame 4
\begin{frame}
  \frametitle{Resultados da Avaliação Empírica}
  Super important text.
\end{frame}

% End section : Cenário 1 - Minimização de Estafetas

%-------------------------------------------------------




%-------------------------------------------------------

% Begin section : Cenário 2 - Maximização dos Lucros

\section{Cenário 2 - Maximização dos Lucros}

% frame 1
\begin{frame}
  \frametitle{Formalização do Problema}
  Super important text.
\end{frame}

% frame 2
\begin{frame}
  \frametitle{Descrição dos Algoritmos}
  Super important text.
\end{frame}

% frame 3
\begin{frame}
  \frametitle{Análise da Complexidade}
  Super important text.
\end{frame}

% frame 4
\begin{frame}
  \frametitle{Resultados da Avaliação Empírica}
  Super important text.
\end{frame}

% End section : Cenário 2 - Maximização dos Lucros

%-------------------------------------------------------




%-------------------------------------------------------

% Begin section : Cenário 3 - Minimização do Tempo de Entrega

\section{Cenário 3 - Minimização do Tempo de Entrega}

% frame 1
\begin{frame}
  \frametitle{Formalização do Problema}
  Super important text.
\end{frame}

% frame 2
\begin{frame}
  \frametitle{Descrição dos Algoritmos}
  Super important text.
\end{frame}

% frame 3
\begin{frame}
  \frametitle{Análise da Complexidade}
  Super important text.
\end{frame}

% frame 4
\begin{frame}
  \frametitle{Resultados da Avaliação Empírica}
  Super important text.
\end{frame}

% End section : Cenário 3 - Minimização do Tempo de Entrega

%-------------------------------------------------------


\end{document}
